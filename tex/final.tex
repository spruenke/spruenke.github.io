\documentclass[12pt]{article}

%%% Nützliche Pakete
\usepackage{graphicx}
\usepackage[T1]{fontenc}
\usepackage[utf8]{inputenc}
\usepackage[english]{babel}
\usepackage{listings}
\usepackage{xcolor}
\usepackage{eso-pic}
\usepackage{mathrsfs}
\usepackage{url}
\usepackage{amssymb}
\usepackage{amsmath}
\usepackage{multirow}
\usepackage{hyperref}
\usepackage{booktabs}
\usepackage{bbm}
\usepackage{longtable}
\usepackage[top=2.5cm,bottom=2.5cm,left=2.5cm,right=2.5cm]{geometry}
\usepackage{lmodern}
\renewcommand*\familydefault{\sfdefault}

%%% Zeilenabstand
\linespread{1.3}

%%% Für Bibliographie
	\usepackage[
		backend=bibtex,
		minbibnames=3, 
		bibstyle=authoryear, 
		citestyle=authoryear, 
		natbib,
		sorting=nyt,
		firstinits=true,
		maxbibnames=99,
		maxcitenames = 2
		]{biblatex}

	\renewbibmacro{in:}{%
	  \ifentrytype{article}{}{\printtext{\bibstring{in}\intitlepunct}}}

	 \DeclareFieldFormat*{citetitle}{#1}
	 \DeclareFieldFormat*{title}{#1}
	 \DeclareNameAlias{sortname}{last-first}
	%%% Füge Bibliographie zur Arbeit hinzu
	\addbibresource{ref1.bib}
	
%%% Bis wieviele Unter-Sektionen soll nummeriert werden	
	\setcounter{secnumdepth}{4}
	\setcounter{tocdepth}{4}
%%% maybe make citation bold

\title{Titel der Arbeit}
\author{Autor der ARbeit}
\date{Hier Datum der Abgabe einfügen}

\pagenumbering{roman}
\begin{document}

		%%%% Titelseite %%%%
		
		\begin{titlepage}
		\pagestyle{empty}
		\begin{center}

		    {\Large{\bf Titel der Arbeit}} \vspace{0.5cm}


		    {\normalsize Bachelor's Thesis submitted\\\vspace{0.5cm}
		    to}\\\vspace{0.5cm}
		    {\normalsize{\bf
			   	 Erstgutachter \\
		 	     Zweitgutachter \\\vspace{0.5cm}
	 		     Advisor: Berater (falls vorhanden)}}\\\vspace{0.5cm}
		    {\normalsize Humboldt-Universit\"at zu Berlin \\
		    School of Business and Economics \\
		    Institut der Fakultät} \vspace{1cm}


		    {\normalsize by \\\vspace{0.5cm}
		    {\bf Autor} \\
		    (Matrikelnummer)} \vspace{1cm}


		    {\normalsize in partial fulfillment of the requirements \\
		    for the degree of \\
		    %%% ggf. hier Master of Science (M.Sc.) einfügen!!
		    {\bf Bachelor of Science (B.Sc.) in Economics} \\ 
		    Berlin, \today} %%% Statt today hier ggf. ein fixes Datum einfügen!
		    %\vfill
		    
		    %%%% Logo der Universität und falls gewünscht des Instituts!
		    \begin{figure}[!b]
		    \centering
		    %%%% Erstes Logo!
		    \begin{minipage}{0.45\textwidth}
		        \centering
		        \includegraphics[scale = 0.9]{hulogo.pdf} % first 		figure itself
		    \end{minipage}\hfill
		    %%% Hier kommt das zweite Logo, falls gewünscht die Prozentzeichen für Kommentare entfernen!
%		    \begin{minipage}{0.45\textwidth}
%		       \centering
%		        \includegraphics[scale = 0.45]{irtglogo1.pdf} % second figure itself
%		    \end{minipage}
		\end{figure}

\end{center}
\end{titlepage}

%%% Ende der Titelseite

%%%
\newpage
\pagestyle{plain}
\pagenumbering{Roman}

\tableofcontents

\newpage

\listoftables

\newpage

\listoffigures

\newpage

\pagenumbering{arabic}

%%% Hier die einzelnen Section-Tex-Files einfügen! Natürlich kann man den Section-Command auch in die einzelnen Files legen, aber so ist etwas übersichtlicher!
\section{Introduction}
\input{Introduction.tex}

\section{Data}
\input{Data.tex}

\section{Methodology}
\input{Methodology.tex}

\section{Empirical Analysis}
\input{Empirical.tex}

\section{Conclusion}
\input{Conclusion.tex}

%%% Bibliographie und Anhang!
\clearpage
\section*{References}
\addcontentsline{toc}{section}{\protect\numberline{}References}%
	\printbibliography[heading = none]
\clearpage
\section*{Appendix A}
\addcontentsline{toc}{section}{\protect\numberline{}Appendix A}%
\input{appendix_a.tex}
\clearpage
\section*{Appendix B}
\addcontentsline{toc}{section}{\protect\numberline{}Appendix B}%
\addtocontents{toc}{\protect\enlargethispage{\baselineskip}}
\input{appendix_b_1.tex}
\clearpage
\input{appendix_b_2.tex}
\clearpage
%%%%%% Eidesstattliche Erklärung!
\section*{Declaration of Honesty}
\input{Honesty.tex}
\end{document}